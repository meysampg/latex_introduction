\chapter{فصل اول}
در این فصل یک نوشته را می‌آوریم و آنرا ادامه می‌دهیم. همچنین می‌توانیم در این بخش یک فصل ایجاد کنیم.

\section{بخش اول}
در این بخش نیز می‌توان چیزهایی نوشت و ادامه داد. بدیهی است که بقیه‌ی امور نیز مانند کارهایی که است که در یک فایل انجام می‌دادیم. در نهایت برای رندر\LTRfootnote{Render} کردن این فایل، فایل \lr{main.tex} را با دستور \lr{xelatex main.tex} در ترمینال و یا با استفاده از ابزارهایی مانند \lr{texMaker} رندر می‌کنیم.

\section{هشدار}
دقت کنید که رندر کردن این فایل (یعنی \lr{chapter01.tex}) موجب بروز خطا می‌شود. همچنین اگر حجم فایل بالا باشد (برای مثال بالای ۳۰۰۰ خط) این امر می‌تواند موجب کرش کردن سیستم‌تان برای چند لحظه شود.

\subsubsection{یک اصطلاح}
اینکه به فرآیند ساخت فایل \lr{PS} یا \lr{PDF} اطلاق چه اصطلاحی مناسب است، همواره ذهن من را مشغول کرده است، بنابر قرارداد بین خودمان، من به آن رندر کردن می‌گویم. توضیحات بیشتر در مورد اصطلاح رندر  را می‌توان در ویکی‌پدیا یا گوگل دید.