\chapter{فرمول چندبخشی}
\section{فرمول چندبخشی هرسطر شماره‌دار}
برای ساختن یک فرمول چندبخشی، از محیط \lr{align} استفاده می‌شود. برای مثال بسط تیلور و جمله‌ی خطایش را برای $x_0\in[a,b]$ می‌نویسیم.
\begin{align}
f(x) &= f(x_0) + (x-x_0)f^\prime(x_0) + \dots + \frac{(x-x_0)^n}{n!}f^{(n)}(x_0) \label{first_line_taylor} \\
	 &+ \frac{(x-x_0)^{n+1}}{(n+1)!}f^{(n+1)}(\xi),\quad \xi\in(a,x_0) \label{taylor_error_term}
\end{align}
و طبیعتا می‌توان به هر کدام از شماره‌ها ارجاع داد. برای مثال \eqref{taylor_error_term} جمله‌ی باقیمانده‌ی بسط تیلور می‌باشد. در صورتی که فقط بخواهیم یکی از سطرها شماره نداشته باشد، از دستور \lr{nonumber} استفاده می‌کنیم. برای مثال در عبارت بالا، سطر اول را بدون شماره چاپ می‌کنیم.
\begin{align}
f(x) &= f(x_0) + (x-x_0)f^\prime(x_0) + \dots + \frac{(x-x_0)^n}{n!}f^{(n)}(x_0) \nonumber \\
	 &+ \frac{(x-x_0)^{n+1}}{(n+1)!}f^{(n+1)}(\xi),\quad \xi\in(a,x_0) \label{taylor_error_term_2}
\end{align}

\section{فرمول چندبخشی با یک شماره‌ی کلی}
برای ساختن یک فرمول چندبخشی که تنها یک شماره ارجاع دارد، از دستور \lr{split} در داخل محیط \lr{equation} استفاده می‌کنیم. برای مثال بسط بالا را با یک شماره به صورت زیر چاپ می‌کنیم.
\begin{equation}
\begin{split}
f(x) &= f(x_0) + (x-x_0)f^\prime(x_0) + \dots + \frac{(x-x_0)^n}{n!}f^{(n)}(x_0) \\
	 &+ \frac{(x-x_0)^{n+1}}{(n+1)!}f^{(n+1)}(\xi),\quad \xi\in(a,x_0)
\end{split}
\end{equation}

\section{فرمول چندبخشی بدون شماره}
برای نمایش یک فرمول چندبخشی، بدون شماره ارجاع، از دستور \lr{align*} استفاده می‌کنیم. دقت کنید که برای استفاده از این دستور، باید بسته‌ی \lr{amsmath} را فراخوانی کرده باشید. برای نمونه داریم:
\begin{align*}
f(x) &= f(x_0) + (x-x_0)f^\prime(x_0) + \dots + \frac{(x-x_0)^n}{n!}f^{(n)}(x_0) \\
	 &+ \frac{(x-x_0)^{n+1}}{(n+1)!}f^{(n+1)}(\xi),\quad \xi\in(a,x_0)
\end{align*}